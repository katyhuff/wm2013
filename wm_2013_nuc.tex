%        File: wm_2013_nuc.tex
%     Created: Mon Aug 13 10:00 AM 2012 C
%
\documentclass[letterpaper]{article}
\usepackage[top=1.0in,bottom=1.0in,left=1.0in,right=1.0in]{geometry}
\usepackage{verbatim}
\usepackage{amssymb}
\usepackage{graphicx}
\usepackage{indentfirst}
\usepackage{longtable}
\usepackage{amsfonts}
\usepackage{amsmath}
\usepackage[usenames]{color}
\usepackage[
naturalnames = true, 
colorlinks = true, 
linkcolor = Black,
anchorcolor = Black,
citecolor = Black,
menucolor = Black,
urlcolor = Blue
]{hyperref}
\def\thesection       {\arabic{section}}
\def\thesubsection     {\thesection.\alph{subsection}}

%% To use the glossaries acronym package, you'll need to define any acronyms you intend to 
%% use. You can define acronyms with \newacronym{label}[acronym]{written out form}
%% To refer to them in the text use \gls{label}
\usepackage[acronym,toc]{glossaries}
\makeglossaries

\author{Kathryn D. Huff
\\ \href{mailto:khuff@anl.gov}{\texttt{khuff@anl.gov}}
}

\date{}
\title{Geologic Nuclide Transport Models in Cyder, A Repository Analysis Software Library}
\begin{document}
\maketitle

\input{acros}


% 1) a descriptive title that will reflect the paper 
%    and presentation content (define all acronyms); 
% 2) a summary of the work conducted, 
%    problem history and your results; 
% 3) all authors contact information including mailing address, email addresses and phone 
%    numbers; and 
% 4) a brief description of the importance of the work 
%    (what problem it addresises/solves) and its application/benefit to others.  


% Abstracts should be between 400 - 800 words and comply with the 
% criteria stated below. Results have shown that brief abstracts that fully and 
% effectively convey the substance and their importance have the highest 
% ranking. Background information irrelevant to an abstract’s merits may 
% dilute the substance and lower the ranking. The evaluation criteria on the 
% “Abstract Rating Form” is posted on the WM website www.wmsym.org
% under the Presenters tab as a guide for authors. Insufficient information 
% for the reviewers may result in session misplacement, lower ranking or 
% abstract rejection.

% Some analytical models have been created for cyder, many levels of detail 
% % Degradation Rate
% % Mixed Cell 
% % Lumped Param
% % OneDimPPM
% Some sensitivity analyses have been conducted to improve the speed 
% % gdsm
% % what params
% %  
% % 
% They've been implemented in a flexible, modular way. 


\section{Introduction and Objectives}

Repository metrics such as necessary repository footprint and peak annual dose 
are affected by heat and radionuclide release characteristics specific to 
variable spent fuel compositions associated with alternative fuel cycles.  For 
this reason, a generic disposal model designed for rapid calculation integration 
with a systems analysis framework is necessary for illuminating performance 
distinctions of candidate repository geologies, designs, and engineering 
components in the context of fuel cycle options. 

The Cyder software library under development integrates with 
the Cyclus computational fuel cycle systems analysis platform in order to 
calculate repository performance metrics with respect to candidate fuel cycle 
options \cite{huff_cyder_2012,huff_cyclus:_2010}. By abstraction of more 
detailed geological nuclide transport models, Cyder aims to capture the dominant 
physics of radionuclide transport phenomena affecting repository performance in 
various geologic media and as a function of arbitrary spent fuel composition.

To support this abstraction effort, sensitivity analyses were performed with 
respect to various key processes and parameters affecting long-term post-closure 
performance of geologic repositories in clay media.  Based on the detailed 
computational Clay \gls{GDSM} developed by the \gls{UFD} campaign 
\cite{clayton_generic_2011}, these results provide an overview of the relative 
importance of processes that affect the repository performance of a generic clay 
disposal concept model. 

\section{Research And Development Overview}

A generic repository model appropriate for systems analysis must emphasize 
modularity and speed. The sensitivity analysis conducted using the clay 
\gls{GDSM} tools captures the dominant physics of detailed repository 
analysis so that abstracted models can be robustly and flexibly implemented in 
Cyder without sacrificing simulation speed.

% used existing gdsms 
Sensitivity analyses were performed using the Clay \gls{GDSM} developed by the 
\gls{UFD} campaign \cite{clayton_generic_2011}.  The Clay \gls{GDSM} is built on 
the GoldSim simulation framework and contaminant transport model.  This 
radionuclide transport toolset simulates chemical and physical attenuation 
processes including radionuclide solubility, dispersion phenomena, and 
reversible sorption \cite{golder_goldsim_2010, golder_goldsim_ct_2010}. Model 
input parameters supporting modeling of these transport processes include 
geometry specifications (e.g. repository depth), geologic material properties 
(e.g. clay porosity), geochemical data (e.g. elemental solubility limits), and 
environmental parameters (e.g. natural system velocity) that affect repository 
performance  \cite{nutt_generic_2009}. 

\section{Accomplishments}

This work has undertaken an analysis strategy to develop a many dimensional 
overview of the key factors in modeled repository performance. The results of 
this work are currently are being used in combination with analytic models of 
solute transport to improve the speed and accuracy of nuclide transport models 
implemented in the Cyder repository analysis library.

Both individual and dual parametric cases were performed and repository 
performance was quantified by radiation dose to a hypothetical receptor.
Individual parameter cases varied a single parameter of interest in 
detail over a broad range of values. Dual parameter cases were 
performed for pairs of parameters expected to exhibit some covariance. For 
each parameter or pair of parameters, forty simulation 
groups varied the parameter or parameters within the range considered. Each 
case and its parametric range are detailed in Table \ref{tab:Cases}. For each 
simulation group, a 100 realization simulation was completed. 

\begin{table}[ht!]
\centering
\footnotesize{
\begin{tabular}{|l|l|l|r|r|}
\multicolumn{5}{c}{\textbf{Simulation Cases}}\\
\hline
\textbf{Case} & \textbf{Parameter} & \textbf{Units} & \textbf{Min. Value} & \textbf{Max. Value}\\
\hline
I     & $D_{eff}$    & $[m^2\cdot s^{-1}]$       & $10^{-8}$    &  $10^{-5}$ \\
      & Inventory              & [MTHM]         & $10^{-4}$    &  $10^1$ \\
\hline
II    & $V_{adv, y}$ & $[m \cdot yr^{-1}]$       & $6.31\times10^{-8}$  &  $6.31\times10^{-4}$ \\
      & $D_{eff}$    & $[m^2\cdot s^{-1}]$       & $10^{-8}$    &  $10^{-5}$ \\
\hline
III   & $S_i$        & $[mol\cdot m^{-3}]$       & $(1\times10^{-9})\langle S_i\rangle $    &  $(5\times10^{10})\langle S_i\rangle $ \\
\hline
IV    & $K_{d,i}$    & $[m^3\cdot kg^{-1}]$       & $(1\times10^{-9})\langle K_{d,i}\rangle $    &  $(5\times10^{10})\langle K_{d,i}\rangle $ \\
\hline
V     & $R_{WFDeg.}$           & $[yr^{-1}]$       & $10^{-9}$    &  $10^{-2}$ \\
      & Inventory              & [MTHM]         & $10^{-4}$    &  $10^1$ \\
\hline 
VI    & $t_{WPFail}$        & $[yr]$         & $10^3$    &  $10^7$ \\
      & $D_{eff}$           & $[m^2\cdot s^{-1}]$       & $10^{-8}$    &  $10^{-5}$ \\
\hline
\end{tabular}
\caption{Each dual and single parameter simulation case had 40 simulation 
groups of 100 realizations each.}
\label{tab:Cases}
}
\end{table}


This work supported completion of the report, \cite{huff_fy12_2012}. 

\bibliographystyle{acm}
\bibliography{bibliography}
\end{document}


