
A base case simulation was conducted as a proof of principle demonstration of 
the modularity and interchangeability of these models. The simplest of the 
contaminant transport models was used to represent the natural and engineered 
barrier components of a repository concept. This concept consisted of a  
saturated clay environment.  In this demonstration, the Cyder open source 
library integrates with the Cyclus computational fuel cycle systems analysis 
platform in order to calculate repository performance metrics with respect to 
candidate fuel cycle options.  Thus, the demonstration illuminates the 
suitability of Cyder's interface for linking to other tools as well as for use 
as a stand-alone radionuclide transport calculation engine.

The demonstration case is an empty software architecture in which to implement 
the physical models. This demonstration has built and tested component module 
loading of models and data, information passing between components represented by 
degradation rate nuclide transport models, and database writing.

Results of unit tests and benchmarking efforts were positive as was a proof of 
principle base case demonstration of the interface between these models. The 
base case demonstration has used the degradation rate to represent nuclide 
transport through waste form, waste package, and buffer components in a generic, 
isotropic, permeable porous geological medium with reducing geochemistry as well 
as the near field. Expected degradation behavior and congruent release was 
observed in unit testing.  

  \begin{figure}[htbp!]
    \begin{center}
      \includegraphics[width=.5\textwidth]{base_case/componentLoading.eps}
      \caption{Waste form, waste package, buffer, and far field components 
        posess transport behavior selected from available transport 
        models, are parameterized by user data, and are loaded modularly 
      into a cohesive framework.}
    \end{center}
  \end{figure}

  \input{base_case/base_case_table}
