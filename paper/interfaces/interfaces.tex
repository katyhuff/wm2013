The interfaces between the models are essential to the understanding of the 
models themselves. The interfaces define boundary conditions in a number of 
forms based on information available internally to the component. 

It is customary to define the combination of molecular diffusion and mechanical
mixing as the dispersion tensor, $D$, such that the mass conservation equation 
becomes:

\begin{align}
  \nabla \left( nD\nabla C \right) - \nabla \left( nv \right) &= 
  \frac{\partial(nC)}{\partial t}
  \label{massbal} \intertext{Adding sorption, by accounting for a change in mass 
  storage,}
  \nabla \left( nD\nabla C \right) - \nabla \left( nv \right)  &= 
  \frac{\partial(nC)}{\partial t}  + \frac{\partial(s\rho_b)}{\partial t} 
  \label{withsorption} \intertext{where}
  s &= \mbox{sorbed phase concentration}\nonumber\\
  \rho_b &= \mbox{ bulk (dry) density }[kg/m^3].\nonumber
\end{align}

If it is assumed that sorption can be approximated as a linear isotherm, 
reversible reaction,

\begin{align}
  \nabla \left( nD\nabla C \right) - \nabla \left( nv \right)  &= 
  \frac{\partial(nC)}{\partial t}  + \frac{\partial(s\rho_b)}{\partial t} 
  \label{linisomasstrans}
  \intertext{where}
  K_d &= \mbox{species distribution coefficient.}\nonumber\\
\end{align}

This becomes 

\begin{align}
  \nabla \left( nD\nabla C \right) - \nabla \left( nv \right)  &= 
  \frac{\partial(nC)}{\partial t}  + \frac{\partial(K_dC\rho_b)}{\partial t} 
  \intertext{which, rearranged, gives}
  \nabla \left( nD\nabla C \right) - \nabla \left( nv \right)  &= 
  \frac{\partial}{\partial t}\left(nC + K_dC\rho_b\right)\\
  \nabla \left( nD\nabla C \right) - \nabla \left( nv \right)  &= 
  \frac{\partial}{\partial t}\left(nC\left(1 + 
  \frac{K_d\rho_b}{n}\right)\right).
  \label{sorptionrearranged}
\end{align}

In equations \eqref{sorptionrearranged} it is clear that the storage term can be 
simplified with a retardation factor, such that if

\begin{align}
  R_f &= \mbox{retardation factor}\\
  &= 1+\frac{\rho_bK_d}{n}\\
  \intertext{then equation \eqref{sorption rearranged} can be written}
  \nabla \left( nD\nabla C \right) - \nabla \left( nv \right) &= 
  R_f\frac{\partial(nC)}{\partial t}    \label{withlinsorption}
\end{align}

For uniform flow, the dispersion tensor, $D$, in equation \ref{uniflow} gives

\begin{align}
  D_x \frac{\partial^2 C}{\partial x^2} +
  D_y \frac{\partial^2 C}{\partial y^2} +
  D_z \frac{\partial^2 C}{\partial z^2} +
  v_x \frac{\partial C}{\partial x}  = R_f \frac{\partial C}{\partial t}.  
  \label{unidirflow}
\end{align}

A special case of uniform flow, no flow, simplifies to the diffusion equation,
\begin{align}
  D_x \frac{\partial^2 C}{\partial x^2} +
  D_y \frac{\partial^2 C}{\partial y^2} +
  D_z \frac{\partial^2 C}{\partial z^2}  = R_f \frac{\partial C}{\partial t} .
  \label{diffusion}
\end{align}

Solutions to these equations can be categorized by their boundary conditions.  
The first, specified-head or Dirichlet type boundary conditions define a specified species 
concentration on some section of the boundary of the representative volume, 

\begin{align}
  C(x,y,z,t) = C_0(x,y,z,t)\hspace{1mm}\mbox{ for } \left( x,y,z \right) \in 
  \Gamma.
\end{align}

The second type, specified-flow or Neumann type boundary conditions describe a full set of 
concentration gradients at the boundary of the domain

\begin{align}
  \frac{\partial C(\vec{r},t)}{\partial r} &= nD\vec{J} \hspace{1mm}\mbox{ for } 
  \vec{r} \in \Gamma.
  \intertext{where}
  \vec{r} &= \mbox{ position vector }\nonumber\\
  \Gamma &= \mbox{ domain boundary }\nonumber\\
  \vec{J} &= \mbox{ solute mass flux } [kg/m^2\cdot s].\nonumber
\end{align}

The third, head-dependent mixed boundary condition or Cauchy type, defines a solute 
flux along a boundary,

\begin{align}
  vC(X,y,z,t) - D_x \frac{\partial C(x,y,z,t}{\partial x} &= 
  vg(x,y,z,t)\hspace{1mm}\mbox{ for } \left( x,y,z \right) \in \Gamma
  \intertext{where}
  g(x,y,z,t) &= \mbox{ arbirary flux profile}
  \intertext{this can also be written}
  -nD_{ij}\frac{\partial C}{\partial x_j}\hat{i} + q_iC\hat{i} &= q_iC_0\hat{i}.
  \intertext{where}
  D_{i,j} &= \hat{i}\mbox{ component of the } [m^2/s]\nonumber\\
  q_i\hat{i} &= \mbox{outward fluid flux} [m/s]\nonumber\\
  \hat{i} &= \mbox{unit vector normal to the surface} [-]\nonumber\\
  C_0 &= \mbox{concentration of the fluid at the boundary} [kg/m^3]\nonumber\\
\end{align}
