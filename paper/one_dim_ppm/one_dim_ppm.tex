\subsubsection*{One Dimensional Permeable Porous Medium Radionuclide Transport 
Model}\label{sec:one_dim_ppm}
Finally, abstraction results informed modifications to the implementation of an 
analytic solution to the one dimensional advection-dispersion equation with 
finite domain and Cauchy and Neumann boundary conditions at the inner and outer 
boundaries, respectively. 

Various solutions to the advection dispersion equation  
\eqref{unidirflow} have been published for both the first and third types of 
boundary conditions. The third, Cauchy type, is mass conservative, and will be 
the primary kind of boundary condition used at the source for this model.

The conceptual model in Figure \ref{fig:1dinf} represents solute transport in 
one dimension with unidirectional flow upward (a conservative assumption) and a 
semi-infinite boundary condition in the positive flow direction. The solution is 
given (Leij et. al, \cite{leij_analytical_1991}) and described below.  

\begin{figure}[h!]
  \begin{center}
    \def\svgwidth{.5\textwidth}
    \input{one_dim_ppm/1dinf.eps_tex}
  \end{center}
  \caption{A one dimensional, semi-infinite model, unidirectional flow,
  solution with Cauchy and Neumann boundary conditions}
  \label{fig:1dinf}
\end{figure}

For the boundary conditions, 
\begin{align}
  -D \frac{\partial C}{\partial z}\big|_{z=0} + v_zc &= \begin{cases}
    v_zC_0  &  \left( 0<t<t_0 \right)\\
    0  &  \left( t>t_0 \right)\\
  \end{cases},\\
  \frac{\partial C}{\partial z}\big|_{z=\infty} &= 0
  \intertext{and the initial condition,}
  C(z,0) &= C_i,
  \label{1dinfBC}
  \intertext{the solution is given as }
  C(z,t) &= 
  \frac{C_0}{4}\bigints_0^t\frac{v}{R}\Lambda_3(\tau)\Gamma_2(\tau)d\tau + 
  \frac{\lambda}{2R}\bigints_0_t\Lambda_4(\tau)
  \intertext{where, for our coordinate system,}
  \Lambda_3(\tau) &= e^{-\frac{\mu\tau}{R}}\Bigg[\sqrt{\frac{R}{\pi D_z\tau}}e^{-\frac{(Rz-v\tau)^2}{4RD_z\tau}} - 
    \frac{v}{2D_z}e^\frac{vz}{D_z}\erfc{\frac{Rz+v\tau}{\sqrt{4RD_z\tau}}}\Bigg]\\
  \Gamma_2(\tau) &= 
      \Bigg[ \erfc{\frac{x-a}{\sqrt{\frac{4D_x\tau}{R}}} } - 
             \erfc{\frac{x+a}{\sqrt{\frac{4D_x\tau}{R}}} } \Bigg]
      \Bigg[ \erfc{\frac{y-b}{\sqrt{\frac{4D_y\tau}{R}}} } -
             \erfc{\frac{y+b}{\sqrt{\frac{4D_y\tau}{R}}} } \Bigg]
  \intertext{and}
  \Lambda_4(\tau) &= e^{-\frac{\mu\tau}{R}}\Bigg[
    \erfc{\frac{v\tau-Rz}{\sqrt{4RD_z\tau}} 
    + \left(1+\frac{v}{D_z}\left(z+\frac{v\tau}{R}\right)\right)e^\frac{vz}{D_z}
    \erfc{\frac{Rz+v\tau}{\sqrt{4RD_z\tau}} 
    - \sqrt{\frac{4v^2\tau}{\pi RD_z}}e^{-\frac{(Rz-v\tau)^2}{4RD_z\tau}}
    \Bigg]\\
\end{align}

We make the simplifying one dimensional assumption that $x=0$ and $y=0$ such 
that $\Gamma_2$ becomes
.


