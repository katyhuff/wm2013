\subsection*{One Dimensional Permeable Porous Medium Radionuclide Transport 
Model}\label{sec:one_dim_ppm}
Finally, abstraction results informed modifications to the implementation of an 
analytic solution to the one dimensional advection-dispersion equation with 
finite domain and Cauchy and Neumann boundary conditions at the inner and outer 
boundaries, respectively. 

Various solutions to the advection dispersion equation in Equation 
\eqref{unidirflow} have been published for both the first and third types of 
boundary conditions. The third, Cauchy type, is mass conservative, and will be 
the primary kind of boundary condition used at the source for this model.


\subsubsection{One Dimensional Semi-Infinite Solution}
The conceptual model in Figure \ref{fig:1dinf} represents solute transport
in one dimension with unidirectional flow and a semi-infinite boundary condition 
in the positive flow direction. 

\vspace{1cm}
\begin{figure}[htbp!]
  \begin{center}
    \def\svgwidth{.5\textwidth}
    \input{one_dim_ppm/1dinf.eps_tex}
  \end{center}
  \caption{Case I, a one dimensional, semi-infinite model.}
  \label{fig:1dinf}
\end{figure}

With the boundary conditions
\begin{align}
  -D \frac{\partial C}{\partial x}\big|_{x=0} + v_xc &= \begin{cases}
    vC_0  &  \left( 0<t<t_0 \right)\\
    0  &  \left( t>t_0 \right)\\
  \end{cases}\\
  \frac{\partial C}{\partial x}\big|_{x=\infty} &= 0
  \intertext{and the initial condition}
  C(x,0) &= C_i,
  \label{1dinfBC}
  \intertext{the solution is given as }
  C(x,t) &=\begin{cases}
    C_i+\left( C_0 - C_i \right)A(x,t) & 0<t<t_0\\
    C_i+\left( C_0 - C_i \right)A(x,t) - C_0A(x,t-t_0) & t>t_0
  \end{cases}
  \intertext{where}
  A(x,t) &= \frac{1}{2}\erfc{ \left[\frac{Rx-vt}{2\sqrt{DRt}}\right] } + \sqrt{ 
  \frac{v^2t}{\pi DR} }e^{-\frac{\left( Rx-vt \right)^2}{4DRt}}.
\end{align}


\subsubsection{One Dimensional Semi-Infinite Solution with Discrete Source}
The conceptual model in Figure \ref{fig:1dinf} represents solute transport
in one dimension with unidirectional flow and a semi-infinite boundary condition 
in the positive flow direction. 

\vspace{1cm}
\begin{figure}[htbp!]
  \begin{center}
    \def\svgwidth{.5\textwidth}
    \input{one_dim_ppm/1dinfwithsrc.eps_tex}
  \end{center}
  \caption{Case II, a one dimensional, semi-infinite model.}
  \label{fig:1dinf}
\end{figure}

With the boundary conditions
\begin{align}
  -D \frac{\partial C}{\partial x}\big|_{x=0} + v_xc &= \begin{cases}
    vC_0  &  \left( 0<t<t_0 \right)\\
    0  &  \left( t>t_0 \right)\\
  \end{cases}\\
  \frac{\partial C}{\partial x}\big|_{x=\infty} &= 0
  \intertext{and the initial condition}
  C(x,0) &= C_i,
  \label{1dinfBC}
  \intertext{the solution is given as }
  C(x,t) &=\begin{cases}
    C_i+\left( C_0 - C_i \right)A(x,t) & 0<t<t_0\\
    C_i+\left( C_0 - C_i \right)A(x,t) - C_0A(x,t-t_0) & t>t_0
  \end{cases}
  \intertext{where}
  A(x,t) &= \frac{1}{2}\erfc{ \left[\frac{Rx-vt}{2\sqrt{DRt}}\right] } + \sqrt{ 
  \frac{v^2t}{\pi DR} }e^{-\frac{\left( Rx-vt \right)^2}{4DRt}}.
\end{align}

\subsubsection{One Dimensional Finite Solution}
The conceptual model in Figure \ref{fig:1dfin} represents solute transport
in one dimension with unidirectional flow and a finite boundary condition in the 
positive flow direction. 

\vspace{1cm}
\begin{figure}[htbp!]
  \begin{center}
    \def\svgwidth{.5\textwidth}
    \input{one_dim_ppm/1dfin.eps_tex}
  \end{center}
  \caption{Case III, a one dimensional, finite model.}
  \label{fig:1dfin}
\end{figure}

With the boundary conditions
\begin{align}
  -D \frac{\partial C}{\partial x}\big|_{x=0} + v_xc &= \begin{cases}
    vC_0  &  \left( 0<t<t_0 \right)\\
    0  &  \left( t>t_0 \right)\\
  \end{cases}\\
  \frac{\partial C}{\partial x}\big|_{x=L} &= 0
  \intertext{and the initial condition}
  C(x,0) &= C_i,
  \label{1dinfBC}
  \intertext{the solution is given as }
  C(x,t) &=\begin{cases}
    C_2+\left( C_1 - C_2 \right)A(x,t) + \left( C_0-C_1 \right)B(x,t)& 0<t<t_0\\
    C_2+\left( C_1 - C_2 \right)A(x,t) + \left( C_0-C_1 \right)B(x,t) - 
    C_0B(x,t-t_0) & 0<t<t_0\\
  \end{cases}
  \intertext{where}
  A(x,t) &= \frac{1}{2}\erfc{\left[ \frac{R(x-x_1)-vt}{2\sqrt{DRt}} \right] } + 
  \sqrt{\frac{v^2t}{\pi DR}} e^{\left[\frac{vx}{D} - \frac{R}{4Dt}(x+x_1 + 
  \frac{vt}{R})^2\right]}\nonumber\\
  & - \frac{1}{2}\left[ 1+ \frac{v(x+x_1)}{D} + 
  \frac{v^2t}{DR}\right]e^{\frac{vx}{D}} \erfc{\left[ 
  \frac{R(x+x_1)+vt}{2\sqrt{DRt}} \right]}\\ B(x,t) &= \frac{1}{2}\erfc{\left[ 
  \frac{Rx-vt}{2\sqrt{DRt}} \right]} + \sqrt{\frac{v^2t}{\pi DR}}e^{-\left[ 
  \frac{(Rx-vt)^2}{4DRt} \right]}\nonumber\\
  & - \frac{1}{2}\left[ 1+ \frac{vx}{D} + \frac{v^2t}{DR}\right]e^{\frac{vx}{D}} 
  \erfc{\left[ \frac{Rx+vt}{2\sqrt{DRt}} \right]}. 
\end{align}
