%        File: wm_2013_nuc.tex
%     Created: Mon Aug 13 10:00 AM 2012 C
%
\documentclass[letterpaper]{article}
\usepackage[top=1.0in,bottom=1.0in,left=1.0in,right=1.0in]{geometry}
\usepackage{amssymb}
\usepackage{graphicx}
%\usepackage{amsfonts}
\usepackage{amsmath}
\usepackage[usenames]{color}
\usepackage[
naturalnames = true, 
colorlinks = true, 
linkcolor = Black,
anchorcolor = Black,
citecolor = Black,
menucolor = Black,
urlcolor = Blue
]{hyperref}
\def\thesection       {\arabic{section}}
\def\thesubsection     {\thesection.\alph{subsection}}
\newcommand{\Cyclus}{\textsc{Cyclus}}

\author{Kathryn D. Huff$^{*}$
\\ $^*$ Argonne National Laboratory, 9700 S. Cass Ave, Argonne, IL. khuff@anl.gov
}

\date{}

% concise yet adequately descriptive title
\title{Hydrologic Nuclide Transport Models in Cyder, A Geologic Disposal Software Library - 13328}
\begin{document}
\maketitle


% 1) a descriptive title that will reflect the paper 
%    and presentation content (define all acronyms); 
% 2) a summary of the work conducted, 
%    problem history and your results; 
% 3) all authors contact information including mailing address, email addresses and phone 
%    numbers; and 
% 4) a brief description of the importance of the work 
%    (what problem it addresises/solves) and its application/benefit to others.  


% Abstracts should be between 400 - 800 words and comply with the 
% criteria stated below. Results have shown that brief abstracts that fully and 
% effectively convey the substance and their importance have the highest 
% ranking. Background information irrelevant to an abstract’s merits may 
% dilute the substance and lower the ranking. The evaluation criteria on the 
% “Abstract Rating Form” is posted on the WM website www.wmsym.org
% under the Presenters tab as a guide for authors. Insufficient information 
% for the reviewers may result in session misplacement, lower ranking or 
% abstract rejection.

% Some analytical models have been created for cyder, many levels of detail 
% % Degradation Rate
% % Mixed Cell 
% % Lumped Param
% % OneDimPPM
% Some sensitivity analyses have been conducted to improve the speed 
% % gdsm
% % what params
% %  
% % 
% They've been implemented in a flexible, modular way. 


% 2. How is the technical problem, issue or program and its importance clearly described?                    
% 3. How are the results or likely results of the investigation/solutions(s) to the problem or 
%    issue described?
%    5 - Very clearly described and supported
%    4 - Above-average
%    3 - Average and somewhat unfocused in specific details
%    2 - Below-average
%    1 - Abstract lacks a clear issue, focus and/or results
% 4. What is the significance of the work/results described?               
%    4 - Significant and widely applicable
%    3 - Applicable to a previously difficult or specific problem or issue resolution
%    2 - Incremental improvement to/status update of an already solved problem      
%    1 –Addresses an already resolved problem     
%    0 - Addresses no known or significant problem or issue by the reviewer
%5. Is the work original and new or an expansion of prior work?             
%    2 – Presents new work/results or new insight
%    1 - Effectively builds on previous work/results
%    0 - Nothing significantly new in work, program, issue resolution or results
%6. How is the information discussed likely to be of interest to attendees? 
%    4 - Will provide an excellent presentation and be a benefit to attendees
%    3 - Will provide an above-average presentation
%    2 - Will result in an average presentation
%    1 - Will be a below-average presentation
%7. Total of the Individual Reviewer’s Scores
\section*{ABSTRACT}

Component level and system level abstraction of detailed computational geologic 
repository models have resulted in four rapid computational models of hydrologic 
radionuclide transport at varying levels of detail. Those models are described, 
as is their implementation in Cyder, a software library of 
interchangeable radionuclide transport models appropriate for representing 
natural and engineered barrier components of generic geology repository 
concepts. A proof of principle demonstration was also conducted in which these models 
were used to represent the natural and engineered barrier components of a 
repository concept in a reducing, homogenous, generic geology. 
This base case demonstrates integration of the Cyder open source library with the Cyclus 
computational fuel cycle systems analysis platform to facilitate calculation of
repository performance metrics with respect to fuel cycle choices.

\section*{INTRODUCTION}
% Provide a summary of the work conducted:
%      Describe the technical problem clearly
%      support it with a method
% SubSection: Motivation
% Provide a brief description of the importance of the work (what problem it 
%   addresses/solves):
Radionuclide containment behavior of a geologic repository is a strong function 
of spent fuel composition, which varies among alternative fuel cycles. For this 
reason, a generic disposal model capable of integration with a systems analysis 
framework is necessary to illuminate performance distinctions of candidate 
repository geologies, designs, and engineering components in the context of fuel 
cycle options. 

A generic repository model appropriate for systems analysis must emphasize 
modularity and speed while providing modeling options at various levels of 
detail. Sensitivity analyses and abstraction efforts conducted to develop the 
models described in this work sought to capture the dominant physics of detailed 
repository behaviors so that abstracted models can be robustly and flexibly 
implemented in the Cyder disposal environment library without sacrificing the 
simulation speed required by the Cyclus fuel cycle simulator.

\section*{MODEL INTERFACES}

The interfaces between the models are essential to the understanding of the 
models themselves. The interfaces define boundary conditions in a number of 
forms based on information available internally to the component. 

It is customary to define the combination of molecular diffusion and mechanical
mixing as the dispersion tensor, $D$, such that the mass conservation equation 
becomes:

  In a saturated, reducing environment, contaminants are transported by 
  dispersion and advection,  
    \begin{align}
      J &= J_{dis} + J_{adv}\nonumber\\
      &= -n(D_{mdis} + \tau D_m)\nabla C + nvC\nonumber\\ 
      &= -nD\nabla C + nvC \nonumber\\ 
      \intertext{which is, for uniform flow}
      &=\left(-nD_{xx} \frac{\partial C}{\partial x}
             + nv_xC \right)\hat{\imath}
             + \left( -nD_{yy} \frac{\partial C}{\partial y}
            \right)\hat{\jmath}
            + \left( -nD_{zz} \frac{\partial C}{\partial z}
            \right)\hat{k},
      \label{unidirflow}
      \intertext{where}
      J_{dif} &= \mbox{ Total Dispersive Mass Flux }[kg/m^2/s]\nonumber\\
      J_{adv} &= \mbox{ Advective Mass Flux }[kg/m^2/s]\nonumber\\
      \tau &= \mbox{ Toruosity }[-] \nonumber\\
      n &= \mbox{ Porosity }[\%] \nonumber\\
      D_m &= \mbox{ Molecular diffusion coefficient }[m^2/s]\nonumber\\
      D_{mdis} &= \mbox{ Coefficient of mechanical dispersivity}[m^2/s]\nonumber\\
      D &= \mbox{ Effective Dispersion Coefficient }[m^2/s]\nonumber\\
      C &= \mbox{ Concentration }[kg/m^3].\\
    \end{align}

Solutions to this equation can be categorized by their boundary conditions and 
those boundary conditions serve as the interfaces between components in the 
Cyder library of nuclide transport models.

  \begin{figure}[htp!]
    \begin{center}
      \def\svgwidth{\textwidth}
      \input{interfaces/flow.eps_tex}
    \end{center}
    \caption{The boundaries between components (in this case, waste form and 
      waste package components) are robust interfaces defined by 
    Source Term, Dirichlet, Neumann, and Cauchy boundary conditions.}
    \label{fig:flow}
  \end{figure}

In addition to a specified source term (the zeroeth type boundary condition, 
perhaps), the first, specified-head or Dirichlet type boundary conditions define a specified species 
concentration on some section of the boundary of the representative volume, 

    \begin{align}
      C(\vec{r},t) = C_0(\vec{r},t)\hspace{1mm}\mbox{ for } \left( \vec{r} \right) \in 
      \Gamma.
    \end{align}

The second type, specified-flow or Neumann type boundary conditions describe a full set of 
concentration gradients at the boundary of the domain

    \begin{align}
      \frac{\partial C(\vec{r},t)}{\partial r} &= nD\vec{J}(t) \hspace{1mm}\mbox{ for } 
      \vec{r} \in \Gamma
      \intertext{where}
      \vec{r} &= \mbox{ position vector }\nonumber\\
      \Gamma &= \mbox{ domain boundary }\nonumber\\
      \vec{J}(t) &= \mbox{ solute mass flux } [kg/m^2\cdot s].\nonumber
    \end{align}
    

The third, head-dependent mixed boundary condition or Cauchy type, defines a solute 
flux along a boundary,

    \begin{align}
      -D\frac{\partial C}{\partial x} + v_xC &= v_xC(\vec{r},t)
      \intertext{where}
      D &= \mbox{ hydrodynamic dispersion coefficient } [m^2/s]\nonumber\\
      v_x &= \mbox{ outward fluid flux} [m/s].\nonumber
    \end{align}  

The spatial concentration throughout the volume is sufficient to fully describe 
implementation of the following nuclide transport models within Cyder. This is 
supported by the implementation in which vertical advective velocity is uniform 
throughout the system and in which parameters such as the dispersion coefficient 
are known for each component. Since this is the case in Cyder, description of 
the Dirichlet condition is sufficient to fully define calculation of the Neumann 
and Cauchy type conditions.




\section*{DESCRIPTION OF THE MODELS}

The results of this work consist of four models that are the product of an 
abstraction effort with more detailed tools.  The analytic models modified by 
abstraction and implemented in Cyder include a degradation rate model, a mixed 
cell model, a response function model, and a one-dimensional solution (Leij and 
Van Genuchten \cite{leij_analytical_1991}) to the advection-dispersion equation.

\input{deg_rate/deg_rate}

\input{mixed_cell/mixed_cell}

\subsection*{Lumped Parameter Radionuclide Model}\label{sec:lumped}

The response function model implemented interchangeable piston flow, 
exponential, and dispersion response functions and was developed by direct 
calibration against the results of the abstraction effort.  


\subsubsection*{One Dimensional Permeable Porous Medium Radionuclide Transport 
Model}\label{sec:one_dim_ppm}
Finally, abstraction results informed modifications to the implementation of an 
analytic solution to the one dimensional advection-dispersion equation with 
finite domain and Cauchy and Neumann boundary conditions at the inner and outer 
boundaries, respectively. 

Various solutions to the advection dispersion equation  
\eqref{unidirflow} have been published for both the first and third types of 
boundary conditions. The third, Cauchy type, is mass conservative, and will be 
the primary kind of boundary condition used at the source for this model.

The conceptual model in Figure \ref{fig:1dinf} represents solute transport in 
one dimension with unidirectional flow upward (a conservative assumption) and a 
semi-infinite boundary condition in the positive flow direction. The solution is 
given (Leij et. al, \cite{leij_analytical_1991}) and described below.  

\begin{figure}[h!]
  \begin{center}
    \def\svgwidth{.5\textwidth}
    \input{one_dim_ppm/1dinf.eps_tex}
  \end{center}
  \caption{A one dimensional, semi-infinite model, unidirectional flow,
  solution with Cauchy and Neumann boundary conditions}
  \label{fig:1dinf}
\end{figure}

For the boundary conditions, 
\begin{align}
  -D \frac{\partial C}{\partial z}\big|_{z=0} + v_zc &= \begin{cases}
    v_zC_0  &  \left( 0<t<t_0 \right)\\
    0  &  \left( t>t_0 \right)\\
  \end{cases},\\
  \frac{\partial C}{\partial z}\big|_{z=\infty} &= 0
  \intertext{and the initial condition,}
  C(z,0) &= C_i,
  \label{1dinfBC}
  \intertext{the solution is given as }
  C(z,t) &= 
  \frac{C_0}{4}\bigints_0^t\frac{v}{R}\Lambda_3(\tau)\Gamma_2(\tau)d\tau + 
  \frac{\lambda}{2R}\bigints_0_t\Lambda_4(\tau)
  \intertext{where, for our coordinate system,}
  \Lambda_3(\tau) &= e^{-\frac{\mu\tau}{R}}\Bigg[\sqrt{\frac{R}{\pi D_z\tau}}e^{-\frac{(Rz-v\tau)^2}{4RD_z\tau}} - 
    \frac{v}{2D_z}e^\frac{vz}{D_z}\erfc{\frac{Rz+v\tau}{\sqrt{4RD_z\tau}}}\Bigg]\\
  \Gamma_2(\tau) &= 
      \Bigg[ \erfc{\frac{x-a}{\sqrt{\frac{4D_x\tau}{R}}} } - 
             \erfc{\frac{x+a}{\sqrt{\frac{4D_x\tau}{R}}} } \Bigg]
      \Bigg[ \erfc{\frac{y-b}{\sqrt{\frac{4D_y\tau}{R}}} } -
             \erfc{\frac{y+b}{\sqrt{\frac{4D_y\tau}{R}}} } \Bigg]
  \intertext{and}
  \Lambda_4(\tau) &= e^{-\frac{\mu\tau}{R}}\Bigg[
    \erfc{\frac{v\tau-Rz}{\sqrt{4RD_z\tau}} 
    + \left(1+\frac{v}{D_z}\left(z+\frac{v\tau}{R}\right)\right)e^\frac{vz}{D_z}
    \erfc{\frac{Rz+v\tau}{\sqrt{4RD_z\tau}} 
    - \sqrt{\frac{4v^2\tau}{\pi RD_z}}e^{-\frac{(Rz-v\tau)^2}{4RD_z\tau}}
    \Bigg]\\
\end{align}

We make the simplifying one dimensional assumption that $x=0$ and $y=0$ such 
that $\Gamma_2$ becomes
.





\section*{DISCUSSION OF THE BASE CASE DEMONSTRATION}

A base case simulation was conducted as a proof of principle demonstration of 
the modularity and interchangeability of these models. The simplest of the 
contaminant transport models was used to represent the natural and engineered 
barrier components of a repository concept. This concept consisted of a  
saturated clay environment.  In this demonstration, the \Cyder open source 
library integrates with the Cyclus computational fuel cycle systems analysis 
platform in order to calculate repository performance metrics with respect to 
candidate fuel cycle options.  Thus, the demonstration illuminates the 
suitability of \Cyder's interface for linking to other tools as well as for use 
as a stand-alone radionuclide transport calculation engine.

The demonstration case is an empty software architecture in which to implement 
the physical models. This demonstration has built and tested component module 
loading of models and data, information passing between components represented by 
degradation rate nuclide transport models, and database writing.

Results of unit tests and benchmarking efforts were positive as was a proof of 
principle base case demonstration of the interface between these models. The 
base case demonstration has used the degradation rate to represent nuclide 
transport through waste form, waste package, and buffer components in a generic, 
isotropic, permeable porous geological medium with reducing geochemistry as well 
as the near field. Expected degradation behavior and congruent release was 
observed in unit testing.  

  \begin{figure}[htbp!]
    \begin{center}
      \includegraphics[width=.5\textwidth]{base_case/componentLoading.eps}
      \caption{Waste form, waste package, buffer, and far field components 
        posess transport behavior selected from available transport 
        models, are parameterized by user data, and are loaded modularly 
      into a cohesive framework.}
    \end{center}
  \end{figure}

  \input{base_case/base_case_table}


\section*{CONCLUDING REMARKS}
The Cyder source code in which these models are implemented as well as 
associated documentation are freely available to interested researchers and 
potential model developers. The application programming interface to this 
software library is intentionally general, facilitating the incorporation of the 
models presented here within external software tools in need of a multicomponent 
repository model.

Furthermore, this work contributes to an expanding ecosystem of computational 
models available for use with the Cyclus fuel cycle simulator. This hydrologic 
nuclide transport library, by virtue of its capability to modularly integrate 
with the Cyclus fuel cycle simulator has laid the foundation for integrated 
disposal option analysis in the context of fuel cycle options. 


\bibliographystyle{plain}
\bibliography{paper}
\end{document}


