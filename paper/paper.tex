%        File: wm_2013_nuc.tex
%     Created: Mon Aug 13 10:00 AM 2012 C
%
\documentclass[letterpaper]{article}
\usepackage[top=1.0in,bottom=1.0in,left=1.0in,right=1.0in]{geometry}
\usepackage{verbatim}
\usepackage{amssymb}
\usepackage{graphicx}
\usepackage{indentfirst}
\usepackage{longtable}
\usepackage{amsfonts}
\usepackage{amsmath}
\usepackage[usenames]{color}
\usepackage[
naturalnames = true, 
colorlinks = true, 
linkcolor = Black,
anchorcolor = Black,
citecolor = Black,
menucolor = Black,
urlcolor = Blue
]{hyperref}
\def\thesection       {\arabic{section}}
\def\thesubsection     {\thesection.\alph{subsection}}

%% To use the glossaries acronym package, you'll need to define any acronyms you intend to 
%% use. You can define acronyms with \newacronym{label}[acronym]{written out form}
%% To refer to them in the text use \gls{label}
\usepackage[acronym,toc]{glossaries}
\makeglossaries

\author{Kathryn D. Huff
\\ \href{mailto:khuff@anl.gov}{\texttt{khuff@anl.gov}}
\\ 9700 S. Cass Ave
}

\date{}

% concise yet adequately descriptive title
\title{Hydrologic Nuclide Transport Models in Cyder, A Geologic Disposal Software Library}
\begin{document}
\maketitle

\newacronym{MIT}{MIT}{the Massachusettes Institute of Technology}
\newacronym{UW}{UW}{University of Wisconsin}
\newacronym{US}{US}{United States}
\newacronym{IAEA}{IAEA}{International Atomic Energy Agency}
\newacronym{SNF}{SNF}{spent nuclear fuel}
\newacronym{HLW}{HLW}{high level waste}
\newacronym{FEHM}{FEHM}{Finite Element Heat and Mass Transfer}
\newacronym{DOE}{DOE}{Department of Energy}
\newacronym{GENIUSv2}{GENIUS}{Global Evaluation of Nuclear Infrastructure Utilization Scenarios, Version 2}
\newacronym{CNERG}{CNERG}{Computational Nuclear Engineering Research Group}
\newacronym{GDSM}{GDSM}{Generic Disposal System Model}
\newacronym{GDSE}{GDSE}{Generic Disposal Sytem Environment}
\newacronym{GPAM}{GPAM}{Generic Performance Asessment Model}
\newacronym{FEPs}{FEPs}{Features, Events, and Processes}
\newacronym{EBS}{EBS}{Engineered Barrier System}
\newacronym{EDZ}{EDZ}{Excavation Disturbed Zone}
\newacronym{YMR}{YMR}{Yucca Mountain Repository Site}
\newacronym{EPA}{EPA}{Environmental Protection Agency}
\newacronym{PEI}{PEI}{Peak Environmental Impact}
\newacronym{VISION}{VISION}{the Verifiable Fuel Cycle Simulation Model}
\newacronym{NUWASTE}{NUWASTE}{Nuclear Waste Assessment System for Technical Evaluation}
\newacronym{NWTRB}{NWTRB}{Nuclear Waste Technical Review Board}
\newacronym{OCRWM}{OCRWM}{Office of Civillian Radioactive Waste Management}
\newacronym{UFD}{UFD}{Used Fuel Disposition}
\newacronym{DYMOND}{DYMOND}{Dynamic Model of Nuclear Development }
\newacronym{DANESS}{DANESS}{Dynamic Analysis of Nuclear Energy System Strategies}
\newacronym{CAFCA}{CAFCA}{ Code for Advanced Fuel Cycles Assessment }
\newacronym{ORION}{ORION}{O..}
\newacronym{NFCSim}{NFCSim}{Nuclear Fuel Cycle Simulator}
\newacronym{COSI}{COSI}{Commelini-Sicard}
\newacronym{FCT}{FCT}{Fuel Cycle Technology}
\newacronym{SWF}{SWF}{Separations and Waste Forms}
\newacronym{FCO}{FCO}{Fuel Cycle Options}
\newacronym{RDD}{RD\&D}{Research Development and Design}
\newacronym{WIPP}{WIPP}{Waste Isolation Pilot Plant}
\newacronym{ANDRA}{ANDRA}{Agence Nationale pour la gestion des D\'echets RAdioactifs, the French National Agency for Radioactive Waste Management}
\newacronym{TSM}{TSM}{Total System Model}
\newacronym{LANL}{LANL}{Los Alamos National Laboratory}
\newacronym{INL}{INL}{Idaho National Laboratory}
\newacronym{ANL}{ANL}{Argonne National Laboratory}
\newacronym{SNL}{SNL}{Sandia National Laboratory}
\newacronym{LBNL}{LBNL}{Lawrence Berkeley National Laboratory}
\newacronym{LLNL}{LLNL}{Lawrence Livermore National Laboratory}
\newacronym{NAGRA}{NAGRA}{National Cooperative for the Disposal of Radioactive Waste}
\newacronym{CUBIT}{CUBIT}{CUBIT Geometry and Mesh Generation Toolkit}
\newacronym{CSNF}{CSNF}{Commercial Spent Nuclear Fuel}
\newacronym{DSNF}{DSNF}{DOE Spent Nuclear Fuel}
\newacronym{MTHM}{MTHM}{Metric Ton of Heavy Metal}
\newacronym{HTGR}{HTGR}{High Temperature Gas Reactor}
\newacronym{TRISO}{TRISO}{Tristructural Isotropic}
\newacronym{MA}{MA}{Minor Actinide}
\newacronym{CEA}{CEA}{Commissariat a l'Energie Atomique et aux Energies Alternatives}
\newacronym{SKB}{SKB}{Svensk Karnbranslehantering AB}
\newacronym{SINDAG}{SINDA{\textbackslash}G}{Systems Improved Numerical Differencing Analyzer $\backslash$ Gaski}
%\newacronym{<++>}{<++>}{<++>}



% 1) a descriptive title that will reflect the paper 
%    and presentation content (define all acronyms); 
% 2) a summary of the work conducted, 
%    problem history and your results; 
% 3) all authors contact information including mailing address, email addresses and phone 
%    numbers; and 
% 4) a brief description of the importance of the work 
%    (what problem it addresises/solves) and its application/benefit to others.  


% Abstracts should be between 400 - 800 words and comply with the 
% criteria stated below. Results have shown that brief abstracts that fully and 
% effectively convey the substance and their importance have the highest 
% ranking. Background information irrelevant to an abstract’s merits may 
% dilute the substance and lower the ranking. The evaluation criteria on the 
% “Abstract Rating Form” is posted on the WM website www.wmsym.org
% under the Presenters tab as a guide for authors. Insufficient information 
% for the reviewers may result in session misplacement, lower ranking or 
% abstract rejection.

% Some analytical models have been created for cyder, many levels of detail 
% % Degradation Rate
% % Mixed Cell 
% % Lumped Param
% % OneDimPPM
% Some sensitivity analyses have been conducted to improve the speed 
% % gdsm
% % what params
% %  
% % 
% They've been implemented in a flexible, modular way. 


% 2. How is the technical problem, issue or program and its importance clearly described?                    
% 3. How are the results or likely results of the investigation/solutions(s) to the problem or 
%    issue described?
%    5 - Very clearly described and supported
%    4 - Above-average
%    3 - Average and somewhat unfocused in specific details
%    2 - Below-average
%    1 - Abstract lacks a clear issue, focus and/or results
% 4. What is the significance of the work/results described?               
%    4 - Significant and widely applicable
%    3 - Applicable to a previously difficult or specific problem or issue resolution
%    2 - Incremental improvement to/status update of an already solved problem      
%    1 –Addresses an already resolved problem     
%    0 - Addresses no known or significant problem or issue by the reviewer
%5. Is the work original and new or an expansion of prior work?             
%    2 – Presents new work/results or new insight
%    1 - Effectively builds on previous work/results
%    0 - Nothing significantly new in work, program, issue resolution or results
%6. How is the information discussed likely to be of interest to attendees? 
%    4 - Will provide an excellent presentation and be a benefit to attendees
%    3 - Will provide an above-average presentation
%    2 - Will result in an average presentation
%    1 - Will be a below-average presentation
%7. Total of the Individual Reviewer’s Scores
\section*{ABSTRACT}

Component level and system level abstraction of detailed computational geologic 
repository models has resulted in four rapid computational models of hydrologic 
radionuclide transport at varying levels of detail. Those models are described, 
as is their implementation in Cyder, a software library of 
interchangeable radionuclide transport models appropriate for representing 
natural and engineered barrier components of generic geology repository 
concepts. A proof of principle demonstration was also conducted in which these models 
were used to represent the natural and engineered barrier components of a 
repository concept in a reducing, homogenous, generic geology. 
In this demonstration, the Cyder open source library integrates with the Cyclus 
computational fuel cycle systems analysis platform in order to calculate 
repository performance metrics with respect to candidate fuel cycle options.  

\section*{INTRODUCTION}
% Provide a summary of the work conducted:
%      Describe the technical problem clearly
%      support it with a method
% SubSection: Motivation
% Provide a brief description of the importance of the work (what problem it 
%   addresses/solves):
Radionuclide containment behavior of a geologic repository is a strong function 
of spent fuel composition, which varies among alternative fuel cycles. For this 
reason, a generic disposal model capable of integration with a systems analysis 
framework is necessary to illuminate performance distinctions of candidate 
repository geologies, designs, and engineering components in the context of fuel 
cycle options. 

A generic repository model appropriate for systems analysis must emphasize 
modularity and speed while providing modeling options at various levels of 
detail. Sensitivity analyses and abstraction efforts conducted to develop the 
models described in this work sought to capture the dominant physics of detailed 
repository behaviors so that abstracted models can be robustly and flexibly 
implemented in the Cyder disposal environment library without sacrificing the 
simulation speed required by the Cyclus fuel cycle simulator.


\section*{DESCRIPTION OF THE MODELS}

The results of this work consist of four models that are the product of an 
abstraction effort with more detailed tools. Results of unit tests and 
benchmarking efforts will be described as will a proof of principle base case 
demonstration of the use of these models. The base case demonstration has used these 
models to represent a generic, isotropic, permeable porous geological medium with 
reducing geochemistry as well as waste form, waste package, and buffer models in 
the near field.

The analytic models modified by abstraction and implemented in Cyder include a 
degradation rate model, a mixed cell model, a response function model, and a 
one-dimensional solution (Leij and Van Genuchten) to the advection-dispersion equation.

\subsection*{Degradation Rate Based Model}

The degradation rate model, simulating the fractional degradation of the material 
containment properties, is the simplest of implemented models and is most 
appropriate for simplistic waste package failure modeling. 

\subsection*{Mixed Cell Model}

Slightly more complex and suited to representing waste form and buffer 
components, the mixed cell model incorporates solubility limited, congruent 
release under the influence of elemental solubility limits, sorption, diffusive 
behavior, and advective behavior. Abstraction results concerning the 
transition between primarily diffusive and primarily advective transport regimes 
were used for benchmarking and to iteratively improve accuracy in the development 
of this model.

\subsection*{Lumped Parameter Response Function Model}

The response function model implemented interchangeable piston flow, 
exponential, and dispersion response functions and was developed by direct 
calibration against the results of the abstraction effort.  

\subsection*{One Dimensional Advection Dispersion Solution with Cauchy Boundary 
Conditions}

Finally, abstraction results informed modifications to the implementation of an 
analytic solution to the one dimensional advection-dispersion equation with 
finite domain and Cauchy and Neumann boundary conditions at the inner and outer 
boundaries, respectively. 

\section*{MODEL INTERFACES}

The interfaces between the models are essential to the understanding of the 
models themselves. The interfaces define boundary conditions in a number of 
forms based on information available internally to the component. 

\section*{BASE CASE DEMONSTRATION}

A base case simulation was conducted as a proof of principle demonstration of 
the modularity and interchangeability of these models. The simplest of the contaminant 
transport models was used to represent the natural and engineered barrier 
components of a repository concept. This concept consisted of a  saturated clay environment.  In 
this demonstration, the Cyder open source library integrates with the Cyclus 
computational fuel cycle systems analysis platform in order to calculate 
repository performance metrics with respect to candidate fuel cycle options.  
Thus, the demonstration illuminates the suitability of Cyder's interface for 
linking to other tools as well as for use as a stand-alone radionuclide 
transport calculation engine.

\section*{RESULTS}
% Provide your results:
%       clearly

\section{Importance to Others}

The Cyder source code in which these models are implemented as well as 
associated documentation are freely available to interested researchers and 
potential model developers. The application programming interface to this 
software library is intentionally general, facilitating the incorporation of the 
models presented here within external software tools in need of a multicomponent 
repository model.

Furthermore, this work contributes to an expanding ecosystem of computational 
models available for use with the Cyclus fuel cycle simulator. This hydrologic 
nuclide transport library, by virtue of its capability to modularly integrate 
with the Cyclus fuel cycle simulator has laid the foundation for integrated 
disposal option analysis in the context of fuel cycle options. 

\end{document}


