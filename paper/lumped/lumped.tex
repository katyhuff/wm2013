\subsection*{Lumped Parameter Radionuclide Model}\label{sec:lumped}

The response function model implemented interchangeable piston flow, 
exponential, and dispersion response functions. The paramters used to describe 
these response functions were developed by direct 
calibration against the results of the abstraction effort.  

For systems in which the flow can be assumed constant, it is possible to model a 
system of volumes as a connected lumped paramter models (Figure 
\ref{fig:lumpedseries}). 


\begin{figure}[htbp!]
  \begin{center}
    \def\svgwidth{.8\textwidth}
    \input{lumpedseries.eps_tex}
  \end{center}
  \caption{A system of volumes can be modeled as lumped parameter models in 
  series.}
  \label{fig:lumpedseries}
\end{figure}

The method by which each lumped parameter component is modeled is according to a 
relationship between the incoming concentration, $C_{in}(t)$, and the outgoing 
concentration, $C_{out}(t)$, \begin{align}
  C_{out}(t) &= \int_{-\infty}^t C_{in}(t')g(t-t')e^{-\lambda(t-t')}dt'
  \label{lumped1}
  \intertext{equivalently}
  C_{out}(t) &= \int_0^\infty C_{in}(t-t')g(t')e^{-\lambda t'}dt'
  \label{lumped2}
  \intertext{where}
  t'  &= \mbox{ time of entry }[s]\nonumber\\
  t-t'  &= \mbox{ transit time }[s]\nonumber\\
  g(t-t')  &= \mbox{ response function, a.k.a. transit time 
  distribution}[-]\nonumber]\\
  \lambda &= \mbox{ radioactive decay constant, 1 to neglect}[s^{-1}]
\end{align}

Selection of the response function is usually based on experimental tracer 
results in the medium at hand. However, some functions used commonly in chemical 
engineering applications include the Piston Flow Model (PFM), \begin{align}
  g(t') &= \delta{(t'-t_t))}
  \intertext{ the Exponential Model (EM) }
  g(t') &= \frac{1}{t_t}e^{-\frac{t'}{t_t}}
  \intertext{ and the Dispersion Model (DM), }
  g(t') &= \frac{\left[ \left( \frac{t\pi t'}{t_t \emph{Pe}} \right) 
  (\frac{1}{t'})e^{- \left( 1- \frac{t'}{t_t}  \right)^2} 
  \right]}{\frac{4t'}{t_t\emph{Pe}}}, \intertext{where}
  \emph{Pe}  &= \mbox{ Peclet number }[-]\nonumber\\
  t_t  &= \mbox{ mean tracer age }[s]\nonumber\\
    &= t_w \mbox{ if there are no stagnant areas}\nonumber\\
  t_w  &= \mbox{ mean residence time of water}[s]\nonumber\\
       &= \frac{V_m}{Q}\nonumber\\
       &= \frac{x}{v_w}\nonumber\\
       &= \frac{xn_e}{v_f}\nonumber
  \intertext{in which}
  V_m  &= \mbox{ mobile water volume}[m^3]\nonumber\\
  Q &= t_w \mbox{ volumetric flow rate }[m^3/s\nonumber\\
  v_w  &= \mbox{ mean water velocity}[m/s]\nonumber\\
  v_f  &= \mbox{ Darcy Flux}[m/s]\nonumber
  \intertext{and}
  n_e  &= \mbox{ effective porosity}[\%]\nonumber.
\end{align}
The latter of these, the Dispersion Model satisfies the one dimensional 
advection-dispersion equation, and is therefore the most physically relevant for 
this application. 

The solutions to these for constant concentration at the source boundary give
\begin{align}
  C(t) &=\begin{cases}
    PFM & C_0e^{\lambda t_t}\\
    EM  & \frac{C_0}{1+\lambda t_t}\\
    DM & \frac{C_0e^{\emph{Pe}\sqrt{\left( 1-(1+\frac{4\lambda 
    t_t}{\emph{Pe}})\right)}}}{2}\\
  \end{cases}
  \label{lumpedsolns}
\end{align}
